Since 1950, the proportion of the world's population living in urban areas has increased from 30\% to approximately 55\%. This trend is projected to continue, with an expected 60\% of the world's population living in cities by 2030 \cite{bravo2018sustainable}. With denser concentrations of people comes denser concentrations of cars and other vehicles, and thus the need for better urban planning, traffic management, and, where necessary, laws restricting the usage of motor-vehicles all together.

Government agencies and local councils often confront these challenges with multiple objectives in mind; improving air quality, improving traffic flow, and improving road safety, to name a few. To understand the effect that various confounding factors have on these measures, urban planners have many tools at their disposal such as air quality surveying, traffic monitoring, and perhaps most importantly, running vast and detailed simulations of cars moving through virtual reconstructions of the area they wish to understand.

For simulations to be representative and reliable, they must look at large areas and simulate the movement of tens of thousands of vehicles. This can be \textit{extremely} computationally expensive, especially if we care about collecting highly granular data at every step of the simulation. 

% This project explores how emulation can be employed to build models of traffic simulators. Our emulators provide a solution that is much faster to run, while also allowing us to analyse the relative importance of input factors affecting the simulation as well as optimise the road network and the vehicles to tackle specific objectives (e.g., reduction of traffic congestion, reduction in CO2 emissions). Taken to production quality, such an emulator could inform time-sensitive decisions that urban planners could take into account when designing the cities of the future.
This project explores how emulation can be employed to build models of traffic simulators. Specifically, we consider the problems of predicting the average amount of CO$_2$ a particular simulation setup will produce, and the average amount of time cars will loose whilst stuck in traffic. Our learned emulators are much faster to run than their underlying simulators. They also allow us to analyse the relative importance of input factors affecting any particular result. Taken to production quality, such an emulator could inform time-sensitive decisions that traffic-management agencies must take to improve congestion, and urban planners must take when designing cities of the future.


The report is structured in the following way: Section~\ref{sec:background} provides background information on traffic simulators and emulation. Section~\ref{sec:simulation} introduces the traffic simulator and its parameters. Section~\ref{sec:exp-setup} discusses the inputs and outputs of the designed emulators. Section~\ref{sec:experimental-design} discusses the experimental design and results of the created emulators. Section~\ref{sec:bay_opt} investigates travel time and emission optimisation using Bayesian Optimisation. Finally, Section~\ref{sec:sensitivity_analysis}, analyses the sensitivity of the emulators with respect to each input parameter. The code used to create and evaluate the emulators can be found at \url{https://github.com/Nicholas-Kastanos/TrafficEmu}